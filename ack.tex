By its nature, data center operation is a combinatory field of very diverse
areas of expertise. Thus, I have had the duty and pleasure of obtaining
knowledge, materials, and skills from a great many individuals from different
departments, institutions, and companies.

From the Department of Computer Science, University of Helsinki I would like
to thank first and foremost the staff of the Computing facilities: Petri
Kutvonen, Pekka Niklander, Ville Hautakangas, Onni Koskinen, Jani Jaakkola,
and Pasi Vettenranta. Without the tenacity of our IT crowd I would have
probably never been able to scavenge all the components required for the
different experiments. Teija Kujala provided me with a splendid and quiet
little corner to read in when I had to recognize that an open office space was
very counterproductive for a solitary researcher. Mikko Rantanen provided his
considerable technical skills derived from his many years in the industry.
Jukka Suomela has repeatedly been very helpful in finding the correct tools
for my trade. Julien Mineraud wrote the tssgpub package that generates the
splash pages and publication lists in this thesis. I am also grateful for the
patience and feedback from members of the Collaborative Networking group.
Finally, Tiina Niklander was a great mentor during my early years at the
department.

The Helsinki Institute for Information Technology (HIIT) was also instrumental
in building our oddball prototypes: Pekka Tonteri, Markus Nuorento, and Sami
Niemimäki were always there to give ideas and feedback when I ran into
trouble. Especially Pekka Tonteri went far beyond the expected while
supporting my endeavours. Without them, building our CAC setups would have
reminded more of constructing a piece of Swedish furniture without schematics,
tools, or the right amount of components.

The University's Technical services also deserve a great many thanks not only
for their construction skills, but also for their understanding and tolerance
of letting us build on the roof of the Exactum building. Nothing much would
have ever been built without Timo Ojanen. Likewise, I'm extremely thankful for
Pirjo Ranta, Markku Hyytiä, and especially Olli Moisio for extending their
help way beyond their normal lines of duty.

The neighbouring Department of Physics formed a beacon of knowledge whenever
my research had to connect with the surrounding real world. Especially Pasi
Aalto and Eki Siivola deserve my thanks. Sampo Smolander was a terrific go-to
guy whenever I had no idea who to talk to. Tomas Lindén and Pekko Metsä from
their IT Department delivered both much needed materials and contacts in the
true spirit of interdepartmental cooperation.

The greenhouse would never have been possible without the support of the Fifth
Dimension project and our cooperation partners: Marja Mesimäki, Gosia Gabrych,
Leena Lindén, Kari Jokinen, Daniel Richterich, Sini Veuro, Ulf Hjelm, Taina
Suonio, and Susanna Lehvävirta. Lassi Remes filled in many of the gaps in my
knowledge of greenhouses, which is to say that the exceptionally good harvest
we got was mostly thanks to him.

My gratitude also goes to members of the industry who lent us their knowledge
and materials at crucial times of the project. From Rittal, Marko Ruokonen,
Jari Peltonen, and Pasi Kinnunen. From Dell, Pekka Vienola. From Windside,
Risto Joutsiniemi and Marja Vähäsarja. From Unicafe, Katja Knuutinen and Miika
Siekkinen. From Halton, Risto Kosonen. From Helen, Juha Sipilä for providing
us with industry contacts we would have not made otherwise. And from CSC, Joni
Virtanen and Peter Jenkins for providing us with both hardware and information
in great quantities.

Members of the Metropoli Bulletin Board System once set me on the path of
system administration. Through our many online discussions, I learned the
basics of critical thinking, logical argumentation, and the tenets of the
hacker ideals. Teppo Oranne was the grand old man of the BBS, and I have tried
to keep in mind his many personal histories from the ICT industry. Johan
Ronkainen has repeatedly taught me that true professional skill comes not
(only) from schools, but from personal dedication and time spent training. 

For their thorough reading and timely comments, I thank my pre-examiners S.
Keshav and Prashant Shenoy. Similarly, Samu Varjonen and Mikko Pitkänen did a
thorough job of reading the thesis, and provided plenty of suggestions and
requests for clarifications. Jussi Kangasharju has remained an excellent
supervisor throughout the research that has lead to this thesis. I could not
have wished for more freedom from my boss and professor.

Last but definitely not least, I would like to thank Laura Langohr, Niko
Välimäki, Riku Katainen, and Panu Luosto from the office room B233. Throughout
our many talks and lunches together, I had the distinct pleasure of learning
how vast our field of computer science truly is. Despite the differences of
our chosen specialties, we often struggled with similar problems, especially
the finer points of \LaTeX.

This work has been supported by the Department of Computer Science, Helsinki
Institute for Information Technology, the Future Internet Graduate School, and
the Nokia Foundation.

\begin{flushright}
In Helsinki, November 4th, 2013\\
\vspace{3em}
Mikko Pervilä\\
\end{flushright}
